
%% following: https://tex.stackexchange.com/questions/38544/glossary-links-color
\makeatletter
\newcommand*{\glsplainhyperlink}[2]{%
  \begingroup%
  \ifglshaslong{\glslabel}%% * does it have long version?
  {\hyperlink{#1}{#2}}%
  {%
    \hypersetup{hidelinks}%
    \hyperlink{#1}{#2}%
  }%% ? used for the first occurrence of abbr/acro (per chapter, and hyper=false is overridden)
  % \hypersetup{hidelinks}%
  % \hyperlink{#1}{#2}%
  \endgroup%
}
\let\@glslink\glsplainhyperlink
\makeatother


\makeglossaries

\setabbreviationstyle[abbreviation]{long-postshort-user}

\glssetcategoryattribute{abbreviation}{nohyperfirst}{true}
\glssetcategoryattribute{general}{glossname}{title}


%% following: https://tex.stackexchange.com/questions/386107/using-package-glossary-and-hyperref-for-acronyms-only-link-from-abbreviation
\renewcommand*{\glsxtruserparen}[2]{%
  \glsxtrfullsep{#2}% space between long part and parenthetical material
  (\glshyperlink[#1]{#2})%
}

\renewcommand*{\glslinkcheckfirsthyperhook}{%
  \ifglsused{\glslabel}%
  {%% * gls label used after first occurrence (configured to be per-chapter)
    \ifglshaslong{\glslabel}%% * does it have long version?
    {\setkeys{glslink}{hyper=true,textformat=normalfont}}%% ? used for abbr/acro that have already been expanded
    {\setkeys{glslink}{hyper=true,textformat=normalfont}}%% ? used for general gls that have already been used once
  }{%% * first use of gls in chapter (hyper=false is overridden, link still included)
    \ifglshaslong{\glslabel}%% * does it have long version?
    {\setkeys{glslink}{hyper=false,textformat=normalfont}}%% ? used for the first occurrence of abbr/acro (per chapter, and hyper=false is overridden)
    {\setkeys{glslink}{hyper=true,textformat=normalfont}}%% ? used for the first occurrence of general gls (hyper=false would disable link)
  }%
}




\newabbreviation%
[description={\textbf{R}equest \textbf{F}or \textbf{C}omments}]
{RFC}
{RFC}
{{R}equest {F}or {C}omments}



\newglossaryentry{pi-calculus}
{
  name={${\pi}$-calculus},
  description={First presented in~\textcite{Milner1999}, the $\pi$-calculus combines elements of the \gls{lambda-calculus}, \glsfirst{CCS} and \glsfirst{CSP} into a single language for describing the behaviour of concurrent processes that exchange information via message-passing.},
  sort={pi-calculus}
}


